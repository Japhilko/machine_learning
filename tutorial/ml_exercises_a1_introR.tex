\PassOptionsToPackage{unicode=true}{hyperref} % options for packages loaded elsewhere
\PassOptionsToPackage{hyphens}{url}
%
\documentclass[
  10pt,
  ignorenonframetext,
]{beamer}
\usepackage{pgfpages}
\setbeamertemplate{caption}[numbered]
\setbeamertemplate{caption label separator}{: }
\setbeamercolor{caption name}{fg=normal text.fg}
\beamertemplatenavigationsymbolsempty
% Prevent slide breaks in the middle of a paragraph:
\widowpenalties 1 10000
\raggedbottom
\setbeamertemplate{part page}{
  \centering
  \begin{beamercolorbox}[sep=16pt,center]{part title}
    \usebeamerfont{part title}\insertpart\par
  \end{beamercolorbox}
}
\setbeamertemplate{section page}{
  \centering
  \begin{beamercolorbox}[sep=12pt,center]{part title}
    \usebeamerfont{section title}\insertsection\par
  \end{beamercolorbox}
}
\setbeamertemplate{subsection page}{
  \centering
  \begin{beamercolorbox}[sep=8pt,center]{part title}
    \usebeamerfont{subsection title}\insertsubsection\par
  \end{beamercolorbox}
}
\AtBeginPart{
  \frame{\partpage}
}
\AtBeginSection{
  \ifbibliography
  \else
    \frame{\sectionpage}
  \fi
}
\AtBeginSubsection{
  \frame{\subsectionpage}
}
\usepackage{lmodern}
\usepackage{amssymb,amsmath}
\usepackage{ifxetex,ifluatex}
\ifnum 0\ifxetex 1\fi\ifluatex 1\fi=0 % if pdftex
  \usepackage[T1]{fontenc}
  \usepackage[utf8]{inputenc}
  \usepackage{textcomp} % provides euro and other symbols
\else % if luatex or xelatex
  \usepackage{unicode-math}
  \defaultfontfeatures{Scale=MatchLowercase}
  \defaultfontfeatures[\rmfamily]{Ligatures=TeX,Scale=1}
\fi
\usetheme[]{Dresden}
\usecolortheme{dolphin}
\usefonttheme{structuresmallcapsserif}
% use upquote if available, for straight quotes in verbatim environments
\IfFileExists{upquote.sty}{\usepackage{upquote}}{}
\IfFileExists{microtype.sty}{% use microtype if available
  \usepackage[]{microtype}
  \UseMicrotypeSet[protrusion]{basicmath} % disable protrusion for tt fonts
}{}
\makeatletter
\@ifundefined{KOMAClassName}{% if non-KOMA class
  \IfFileExists{parskip.sty}{%
    \usepackage{parskip}
  }{% else
    \setlength{\parindent}{0pt}
    \setlength{\parskip}{6pt plus 2pt minus 1pt}}
}{% if KOMA class
  \KOMAoptions{parskip=half}}
\makeatother
\usepackage{xcolor}
\IfFileExists{xurl.sty}{\usepackage{xurl}}{} % add URL line breaks if available
\IfFileExists{bookmark.sty}{\usepackage{bookmark}}{\usepackage{hyperref}}
\hypersetup{
  pdftitle={ML exercises - basics R},
  pdfauthor={Jan-Philipp Kolb},
  pdfborder={0 0 0},
  breaklinks=true}
\urlstyle{same}  % don't use monospace font for urls
\newif\ifbibliography
\usepackage{color}
\usepackage{fancyvrb}
\newcommand{\VerbBar}{|}
\newcommand{\VERB}{\Verb[commandchars=\\\{\}]}
\DefineVerbatimEnvironment{Highlighting}{Verbatim}{commandchars=\\\{\}}
% Add ',fontsize=\small' for more characters per line
\newenvironment{Shaded}{}{}
\newcommand{\AlertTok}[1]{\textcolor[rgb]{1.00,0.00,0.00}{#1}}
\newcommand{\AnnotationTok}[1]{\textcolor[rgb]{0.00,0.50,0.00}{#1}}
\newcommand{\AttributeTok}[1]{#1}
\newcommand{\BaseNTok}[1]{#1}
\newcommand{\BuiltInTok}[1]{#1}
\newcommand{\CharTok}[1]{\textcolor[rgb]{0.00,0.50,0.50}{#1}}
\newcommand{\CommentTok}[1]{\textcolor[rgb]{0.00,0.50,0.00}{#1}}
\newcommand{\CommentVarTok}[1]{\textcolor[rgb]{0.00,0.50,0.00}{#1}}
\newcommand{\ConstantTok}[1]{#1}
\newcommand{\ControlFlowTok}[1]{\textcolor[rgb]{0.00,0.00,1.00}{#1}}
\newcommand{\DataTypeTok}[1]{#1}
\newcommand{\DecValTok}[1]{#1}
\newcommand{\DocumentationTok}[1]{\textcolor[rgb]{0.00,0.50,0.00}{#1}}
\newcommand{\ErrorTok}[1]{\textcolor[rgb]{1.00,0.00,0.00}{\textbf{#1}}}
\newcommand{\ExtensionTok}[1]{#1}
\newcommand{\FloatTok}[1]{#1}
\newcommand{\FunctionTok}[1]{#1}
\newcommand{\ImportTok}[1]{#1}
\newcommand{\InformationTok}[1]{\textcolor[rgb]{0.00,0.50,0.00}{#1}}
\newcommand{\KeywordTok}[1]{\textcolor[rgb]{0.00,0.00,1.00}{#1}}
\newcommand{\NormalTok}[1]{#1}
\newcommand{\OperatorTok}[1]{#1}
\newcommand{\OtherTok}[1]{\textcolor[rgb]{1.00,0.25,0.00}{#1}}
\newcommand{\PreprocessorTok}[1]{\textcolor[rgb]{1.00,0.25,0.00}{#1}}
\newcommand{\RegionMarkerTok}[1]{#1}
\newcommand{\SpecialCharTok}[1]{\textcolor[rgb]{0.00,0.50,0.50}{#1}}
\newcommand{\SpecialStringTok}[1]{\textcolor[rgb]{0.00,0.50,0.50}{#1}}
\newcommand{\StringTok}[1]{\textcolor[rgb]{0.00,0.50,0.50}{#1}}
\newcommand{\VariableTok}[1]{#1}
\newcommand{\VerbatimStringTok}[1]{\textcolor[rgb]{0.00,0.50,0.50}{#1}}
\newcommand{\WarningTok}[1]{\textcolor[rgb]{0.00,0.50,0.00}{\textbf{#1}}}
\setlength{\emergencystretch}{3em}  % prevent overfull lines
\providecommand{\tightlist}{%
  \setlength{\itemsep}{0pt}\setlength{\parskip}{0pt}}
\setcounter{secnumdepth}{-2}

% set default figure placement to htbp
\makeatletter
\def\fps@figure{htbp}
\makeatother


\title{ML exercises - basics R}
\author{Jan-Philipp Kolb}
\date{03 Juni, 2019}

\begin{document}
\frame{\titlepage}

\begin{frame}{Exercise: Find R-packages}
\protect\hypertarget{exercise-find-r-packages}{}

Go to \url{https://cran.r-project.org/} and search for packages that can
be used:

\begin{enumerate}
[1)]
\tightlist
\item
  to reduce overfitting
\item
  for regression trees
\item
  for gradient boosting
\item
  for neural networks
\item
  for clustering
\end{enumerate}

\end{frame}

\begin{frame}[fragile]{Solution: Find R-packages}
\protect\hypertarget{solution-find-r-packages}{}

\begin{Shaded}
\begin{Highlighting}[]
\KeywordTok{install.packages}\NormalTok{(}\StringTok{"glmnet"}\NormalTok{) }\CommentTok{#1)}
\KeywordTok{install.packages}\NormalTok{(}\StringTok{"rpart"}\NormalTok{) }\CommentTok{#2)}
\KeywordTok{install.packages}\NormalTok{(}\StringTok{"gbm"}\NormalTok{) }\CommentTok{#3)}
\KeywordTok{install.packages}\NormalTok{(}\StringTok{"neuralnet"}\NormalTok{) }\CommentTok{#4)}
\KeywordTok{install.packages}\NormalTok{(}\StringTok{"kknn"}\NormalTok{) }\CommentTok{#5)}
\end{Highlighting}
\end{Shaded}

\end{frame}

\begin{frame}[fragile]{Exercise: load built-in data}
\protect\hypertarget{exercise-load-built-in-data}{}

\begin{block}{Load the the built-in dataset \texttt{swiss}}

\begin{enumerate}
[1)]
\tightlist
\item
  How many observations and variables are available?
\item
  What is the scale level of the variables?
\end{enumerate}

\end{block}

\begin{block}{Interactive data table}

\begin{enumerate}
[1)]
\setcounter{enumi}{2}
\tightlist
\item
  Create an interactive data table
\end{enumerate}

\end{block}

\end{frame}

\begin{frame}[fragile]{Solution: load built-in data}
\protect\hypertarget{solution-load-built-in-data}{}

\begin{Shaded}
\begin{Highlighting}[]
\CommentTok{# 1)}
\KeywordTok{data}\NormalTok{(swiss) }
\KeywordTok{dim}\NormalTok{(swiss) }
\end{Highlighting}
\end{Shaded}

\begin{verbatim}
## [1] 47  6
\end{verbatim}

\begin{Shaded}
\begin{Highlighting}[]
\KeywordTok{str}\NormalTok{(swiss) }
\end{Highlighting}
\end{Shaded}

\begin{verbatim}
## 'data.frame':    47 obs. of  6 variables:
##  $ Fertility       : num  80.2 83.1 92.5 85.8 76.9 76.1 83.8 92.4 82.4 82.9 ...
##  $ Agriculture     : num  17 45.1 39.7 36.5 43.5 35.3 70.2 67.8 53.3 45.2 ...
##  $ Examination     : int  15 6 5 12 17 9 16 14 12 16 ...
##  $ Education       : int  12 9 5 7 15 7 7 8 7 13 ...
##  $ Catholic        : num  9.96 84.84 93.4 33.77 5.16 ...
##  $ Infant.Mortality: num  22.2 22.2 20.2 20.3 20.6 26.6 23.6 24.9 21 24.4 ...
\end{verbatim}

\begin{Shaded}
\begin{Highlighting}[]
\CommentTok{# 2)}
\NormalTok{DT}\OperatorTok{::}\KeywordTok{datatable}\NormalTok{(swiss)}
\end{Highlighting}
\end{Shaded}

\end{frame}

\begin{frame}[fragile]{\href{https://www.datacamp.com/community/tutorials/pipe-r-tutorial}{Exercise}:
random numbers}
\protect\hypertarget{exercise-random-numbers}{}

\begin{enumerate}
[1)]
\tightlist
\item
  Draw 8 random numbers from the uniform distribution and save them in a
  vector \texttt{x}
\item
  Compute the logarithm of \texttt{x}, return suitably lagged and
  iterated differences,
\item
  compute the exponential function and round the result
\end{enumerate}

\begin{verbatim}
## [1] 6.4 0.6 2.2 0.7 1.5 0.8 1.0
\end{verbatim}

\end{frame}

\begin{frame}[fragile]{Solution: random numbers}
\protect\hypertarget{solution-random-numbers}{}

\begin{verbatim}
## [1] 1.0 1.1 0.3 0.5 2.1 0.6 6.2
\end{verbatim}

\end{frame}

\end{document}
